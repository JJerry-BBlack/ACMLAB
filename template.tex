\documentclass[twocolumn,a4paper]{article}  %两栏,A4大小

\usepackage{xeCJK} % 中文支持
\setCJKmainfont{KaiTi} % 设置中文字体

\usepackage{amsmath} % 数学公式支持
% \let\over\frac % 替换\over命令为\frac

\usepackage{listings,xcolor}  %插入代码
\usepackage[landscape]{geometry} % 设置页边距
\usepackage{fontspec}
\usepackage{graphicx}
\usepackage{fancyhdr} % 自定义页眉页脚
\usepackage{calligra}
\usepackage{epstopdf}
\usepackage[hidelinks]{hyperref}
\usepackage{subfig}

\setsansfont{Consolas} % 设置英文字体
\setmonofont[Mapping={}]{Maple Mono} % 英文引号之类的正常显示,相当于设置英文字体
\geometry{left=1cm,right=1cm,top=2cm,bottom=0.5cm} % 页边距
\setlength{\columnsep}{30pt}  %两栏之间的间距大小
% \setlength\columnseprule{0.4pt} % 分割线

% 页眉、页脚设置
\pagestyle{fancy}
% \lhead{CUMTB}
\lhead{\CJKfamily{hei} ACM-ICPC Code Template}
\chead{}
% \rhead{Page \thepage}
\rhead{\CJKfamily{hei} 第 \thepage 页}
\lfoot{} 
\cfoot{}
\rfoot{}
\renewcommand{\headrulewidth}{0.4pt} 
\renewcommand{\footrulewidth}{0.4pt}

% 代码格式设置
\lstset{
	language    = c++,
	numbers     = left,
	numberstyle = \tiny,
	breaklines  = true,
	captionpos  = b,
	tabsize     = 4,
	frame       = shadowbox,
	columns     = fullflexible,
	commentstyle = \color[RGB]{0,128,0}\textit,
	keywordstyle = \color[RGB]{0,0,255}\textit,
	basicstyle   = \small\ttfamily,
	stringstyle  = \color[RGB]{148,0,209}\ttfamily,
	rulesepcolor = \color{red!20!green!20!blue!20},
	showstringspaces = false,
}

\title{\CJKfamily{hei} \bfseries ACM-ICPC Code Template}
\author{Jerry Black}
\renewcommand{\today}{\number\year 年 \number\month 月 \number\day 日}

\begin{document}\small
	
\begin{titlepage}
\newcommand{\HRule}{\rule{\linewidth}{0.5mm}}
\begin{figure}[htbp]
	\centering
	\subfloat{
		\includegraphics[height=5.5cm]{pic/ACM.png}
	}
	\hfil
	\subfloat{
		\includegraphics[height=5.5cm]{pic/logo.png}
	}
\end{figure}
\centering
\quad\\[2cm]
\textsl{\Large Zhejiang Normal University}\\[0.5cm] 
\textsl{\large School of Mathematical Sciences}\\[0.5cm] 
\makeatletter
\HRule \\[0.4cm]
{ \huge \bfseries \@title}\\[0.4cm] 
\HRule \\[1.5cm]
\begin{minipage}{0.4\textwidth}
	\begin{flushleft} \large
		\emph{Author:}\\
		\@author 
	\end{flushleft}
\end{minipage}
~
\begin{minipage}{0.4\textwidth}
	\begin{flushright} \large
		\emph{Date:} \\
		\textup{\today}
	\end{flushright}
\end{minipage}\\[3cm]
\makeatother
\vfill
\end{titlepage}

\newpage
\pagestyle{empty}
\renewcommand{\contentsname}{Contest}
\tableofcontents %生成目录
\newpage\clearpage
\newpage
\pagestyle{fancy}
\setcounter{page}{1}   %从当前页开始计算页数
%% ====================================================================
\section{Graph}
\subsection{Dijkstra}
\lstinputlisting{Graph/Dijkstra.cpp}
\subsection{匈牙利算法}
\lstinputlisting{Graph/xiongyali.cpp}
\subsection{Hopcroft\_Karp(匈牙利算法优化)}
\lstinputlisting{Graph/HopcroftKarp.cpp}
\subsection{二分图染色+拓扑排序}
\lstinputlisting{Graph/BipartiteColoring&TopologicalSorting.cpp}
\subsection{网络流}
\subsubsection{最大流}
\lstinputlisting{Graph/Dinic.cpp}
\subsubsection{KM}
\lstinputlisting{Graph/KM.cpp}
\subsubsection{最小费用最大流}
\lstinputlisting{Graph/MinCostMaxFlow.cpp}
\subsubsection{最大费用可行流}
\lstinputlisting{Graph/MaxCostFeasibleFlow.cpp}
\subsection{连通分量}
\subsubsection{强连通分量}
\lstinputlisting{Graph/SCC.cpp}
\subsubsection{边双连通分量}
\lstinputlisting{Graph/e-DCC.cpp}
\subsubsection{点双连通分量}
\lstinputlisting{Graph/p-DCC.cpp}
\subsubsection{割点}
\lstinputlisting{Graph/cutvertex.cpp}
\subsubsection{割边}
\lstinputlisting{Graph/bridge.cpp}
\subsubsection{圆方树}
\lstinputlisting{Graph/block-forest.cpp}
\subsection{2-SAT}
\lstinputlisting{Graph/twosat.cpp}
\subsection{虚树}
\subsection{树分治}
\subsubsection{点分治}
\lstinputlisting{Graph/pointdivide.cpp}
\subsubsection{点分树}
\lstinputlisting{Graph/pointdividetree.cpp}
\lstinputlisting{Graph/virtual-tree.cpp}
\subsection{异或最小生成树}
\lstinputlisting{Graph/XorMST.cpp}
%% ===================================================================
\newpage
\section{Data Structure}
\subsection{种类并查集}
\lstinputlisting{DataStructure/KindDSU.cpp}
\subsection{二维树状数组}
\subsubsection{单点修改,区间查询}
\lstinputlisting{DataStructure/2DFenwickTree1.cpp}
\subsubsection{区间修改,单点查询}
\lstinputlisting{DataStructure/2DFenwickTree2.cpp}
\subsubsection{区间修改,区间查询}
\lstinputlisting{DataStructure/2DFenwickTree3.cpp}
\subsection{线段树}
\subsubsection{维护等差数列}
\lstinputlisting{DataStructure/SegTree1.cpp}
\subsubsection{有加有乘}
\lstinputlisting{DataStructure/SegTree2.cpp}
\subsubsection{最大子段和}
\lstinputlisting{DataStructure/SegTree3.cpp}
\subsection{非递归版本线段树}
\lstinputlisting{DataStructure/Lazy_Segtree.cpp}
\subsubsection{用法1:}
\lstinputlisting{DataStructure/Lazy_Segtree_Usage1.cpp}
\subsubsection{用法2:}
\lstinputlisting{DataStructure/Lazy_Segtree_Usage2.cpp}
\subsection{动态开点线段树}
\lstinputlisting{DataStructure/DynamicSegTree.cpp}
\subsection{Segment Tree Beats}
\subsubsection{区间取min 、max  $O(nlogn)$}
\lstinputlisting{DataStructure/SegTreeBeats1.cpp}
\subsubsection{区间取min 、max +区间加减 $O(nlog^2n)$}
\lstinputlisting{DataStructure/SegTreeBeats2.cpp}
\subsection{可持久化线段树}
\subsubsection{静态版本,权值线段树}
\lstinputlisting{DataStructure/SustainableSegTree1.cpp}
\subsubsection{动态版本,支持修改}
\lstinputlisting{DataStructure/SustainableSegTree2.cpp}
\subsection{线段树合并\&分裂}
\lstinputlisting{DataStructure/SegTreeMerge&Split.cpp}
\subsection{分块}
\lstinputlisting{DataStructure/Sqrt.cpp}
\subsection{珂朵莉树}
\lstinputlisting{DataStructure/OldDriversTree.cpp}
\subsection{莫队}
\subsubsection{普通莫队}
\lstinputlisting{DataStructure/Mo1.cpp}
\subsubsection{带修莫队}
\lstinputlisting{DataStructure/Mo2.cpp}
\subsubsection{树上莫队}
\lstinputlisting{DataStructure/Mo3.cpp}
\subsubsection{回滚莫队}
\lstinputlisting{DataStructure/Mo4.cpp}
\lstinputlisting{DataStructure/Mo5.cpp}
\subsection{CDQ分治}
\subsubsection{二维}
\lstinputlisting{DataStructure/2DCDQ.cpp}
\subsubsection{三维}
\lstinputlisting{DataStructure/3DCDQ.cpp}
\subsection{链式前向星}
\lstinputlisting{DataStructure/ChainForwardStar.cpp}
\subsection{Kruskal重构树}
\lstinputlisting{DataStructure/KruskalRebuildTree.cpp}
\subsection{树链剖分}
\lstinputlisting{DataStructure/poufen.cpp}
\subsection{ST表}
\lstinputlisting{DataStructure/SparseTable.cpp}
\subsection{DSU on Tree}
\lstinputlisting{DataStructure/DsuOnTree.cpp}
\subsection{线性基}
\lstinputlisting{DataStructure/LinearBase.cpp}
\subsection{平衡树}
\subsubsection{替罪羊树}
\lstinputlisting{DataStructure/ScapegoatTree.cpp}
\subsubsection{FHQ\_Treap}
\lstinputlisting{DataStructure/FHQTreap1.cpp}
\lstinputlisting{DataStructure/FHQTreap2.cpp}
\subsubsection{Splay}
\lstinputlisting{DataStructure/Splay.cpp}
\subsection{LCT}
\lstinputlisting{DataStructure/LCT.cpp}
%% ===================================================================
\newpage
\section{String}
\subsection{KMP}
\lstinputlisting{String/KMP.cpp}
\subsection{马拉车}
\lstinputlisting{String/manacher.cpp}
\subsection{哈希}
\subsubsection{一维}
\lstinputlisting{String/Hash1.cpp}
\subsubsection{二维}
\lstinputlisting{String/Hash2.cpp}
\subsection{Trie树}
\lstinputlisting{String/Trie.cpp}
\subsection{AC自动机}
\lstinputlisting{String/ACAM.cpp}
\subsection{回文自动机}
\lstinputlisting{String/PAM1.cpp}
\subsection{广义回文自动机}
\lstinputlisting{String/PAM2.cpp}
\subsection{后缀数组(SA)}
\lstinputlisting{String/SA.cpp}
\subsection{后缀数组(SA-IS)}
\lstinputlisting{String/SA-IS.cpp}
\subsection{后缀自动机}
\lstinputlisting{String/SAM1.cpp}
\subsection{广义后缀自动机}
\lstinputlisting{String/SAM2.cpp}
%% ===================================================================
\newpage
\section{Mathematics}
\subsection{快速乘}
\lstinputlisting{Math/FastMul.cpp}
\subsection{欧几里得算法}
\lstinputlisting{Math/Gcd.cpp}
\subsection{拓展欧几里得算法}
\lstinputlisting{Math/exGcd.cpp}
\subsection{BSGS}
\lstinputlisting{Math/BSGS.cpp}
\subsection{埃氏筛}
\lstinputlisting{Math/EratosthenesSieve.cpp}
\subsection{欧拉筛}
\lstinputlisting{Math/EularSieve.cpp}
\subsection{米勒罗宾素性测试\&Pollard rho算法}
\lstinputlisting{Math/MillerRobin&PollardRho.cpp}
\subsection{Stein算法(大整数的GCD)}
\lstinputlisting{Math/Stein.cpp}
\subsection{逆元}
\lstinputlisting{Math/Inv1.cpp}
\lstinputlisting{Math/Inv2.cpp}
\subsection{欧拉函数}
\lstinputlisting{Math/Eular1.cpp}
\lstinputlisting{Math/Eular2.cpp}
\subsection{欧拉降幂}
$$
	a^b\mod c=
	\begin{cases}
		a^{b\mod \varphi(c)}, &\gcd(a,c)=1\\
		a^b, &\gcd(a,c)\ne1,b<\varphi(m)\\
		a^{b\mod \varphi(c)+\varphi(c)}, &\gcd(a,c)\ne1,b\ge\varphi(m)
	\end{cases}\mod c
$$
\subsection{拓展中国剩余定理}
\lstinputlisting{Math/exCRT.cpp}
\subsection{中国剩余定理}
\lstinputlisting{Math/CRT.cpp}
\subsection{二次剩余}
\lstinputlisting{Math/Cipolla.cpp}
\subsection{拓展欧拉定理}
\lstinputlisting{Math/EularDropPow.cpp}
\subsection{组合数}
\lstinputlisting{Math/Binom1.cpp}
\lstinputlisting{Math/Binom2.cpp}
\subsection{卢卡斯定理}
\lstinputlisting{Math/Lucas.cpp}
\subsection{拓展卢卡斯定理}
\lstinputlisting{Math/exLucas.cpp}
\subsection{积性函数}
\lstinputlisting{Math/MathFunction1.cpp}
\lstinputlisting{Math/MathFunction2.cpp}
\subsection{杜教筛}
$$
\begin{aligned}
sum(n)&=\sum_{i=1}^{n}f(i)\\
\sum_{i=1}^n h(i)=\sum_{i=1}^{n}(f*g)(i)&=\sum_{i=1}^n \sum_{d\mid i}g(d)f\left({\frac{i}{d}}\right)=
\sum_{d=1}^n g(d)\sum_{i=1}^{\left\lfloor{\frac{n}{d}}\right\rfloor}=
\sum_{d=1}^n g(d)sum\left(\left\lfloor{\frac{n}{d}}\right\rfloor\right)\\
\Rightarrow g(1)sum(n)&=\sum_{i=1}^n h(i)-\sum_{i=2}^n g(i)sum\left(\left\lfloor{\frac{n}{i}}\right\rfloor\right)
\end{aligned}
$$
\lstinputlisting{Math/DYHSieve.cpp}
\subsection{Min\_25筛}
To Be Studied...
\lstinputlisting{Math/Min25Sieve.cpp}
\subsection{莫比乌斯函数}
\lstinputlisting{Math/Mobios.cpp}
\subsection{快速数论变换}
\lstinputlisting{Math/NTT.cpp}
\subsection{FWT}
\lstinputlisting{Math/FWT.cpp}
\subsection{BM}
\lstinputlisting{Math/BM.cpp}
\subsection{矩阵求逆}
\lstinputlisting{Math/MatrixInv.cpp}
\subsection{矩阵乘法\&快速幂}
\lstinputlisting{Math/Matrix.cpp}
\subsection{整除分块}
\lstinputlisting{Math/Divide.cpp}
\subsection{自适应辛普森积分}
$$
\int_a^bf(x)\mathrm{d}x\approx{\frac{b-a}{6}}\left[f(a)+4\times f\left({\frac{a+b}{2}}\right)+f(b)\right]
$$
\lstinputlisting{Math/Simpson.cpp}
\subsection{最小二乘法}
给定 $n$ 个点,求一条直线 $y=kx+b$ 最匹配(写出方差式子求偏导)
$$
\begin{cases}
	\displaystyle \Sigma xk+nb-\Sigma y=0\\
	\displaystyle \Sigma x^2k+\Sigma xb-\Sigma xy=0
\end{cases}
\rightarrow
\begin{cases}
	\displaystyle k={\frac{-\Sigma x\Sigma y+n\Sigma xy}{\Sigma x\Sigma x-\Sigma x^2n}}\\
	\displaystyle b={\frac{-\Sigma x^2\Sigma y+\Sigma x\Sigma xy}{\Sigma x^2n-\Sigma x\Sigma x}}
\end{cases}
$$
\lstinputlisting{Math/OrdinaryLeastSquares.cpp}
\subsection{高精度}
\lstinputlisting{Math/BigInt.cpp}
%% ===================================================================
\newpage
\section{Dynamic Programming}
\subsection{最长公共子序列}
\lstinputlisting{DP/LCS.cpp}
\subsection{背包}
\subsubsection{01背包}
\lstinputlisting{DP/01BackPack.cpp}
\subsubsection{完全背包}
\lstinputlisting{DP/CompleteBackPack.cpp}
\subsubsection{多重背包}
\lstinputlisting{DP/MultipleBackPack.cpp}
\subsubsection{多种背包混合问题}
\lstinputlisting{DP/MixedBackPack.cpp}
\subsubsection{二维费用的背包问题}
\lstinputlisting{DP/2DFeaBackPack.cpp}
\subsubsection{分组的背包问题}
\lstinputlisting{DP/GroupBackPack.cpp}
\subsubsection{有依赖的背包问题}
\lstinputlisting{DP/DependentBackPack.cpp}
\subsection{状压dp}
\lstinputlisting{DP/TSP.cpp}
\subsection{数位dp}
\lstinputlisting{DP/DigitalDP.cpp}
\subsection{带障碍的路径计数}
\lstinputlisting{DP/RoadCountingWithBlocks.cpp}
%% ===================================================================
\newpage
\section{Geometry}
\subsection{二维}
\lstinputlisting{Geometry/2D.cpp}
\subsection{三维}
\lstinputlisting{Geometry/3D.cpp}
%% ===================================================================
\newpage
\section{Others}
\subsection{头文件}
\lstinputlisting{Others/Head.cpp}
\subsection{吸氧大法}
\lstinputlisting{Others/Oxygen.cpp}
\subsection{迭代器}
\lstinputlisting{Others/Iterator.cpp}
\subsection{lambda函数}
\lstinputlisting{Others/Lambda.cpp}
\subsection{结构体}
\lstinputlisting{Others/Struct.cpp}
\subsection{快读快写}
\lstinputlisting{Others/FastIO1.cpp}
\lstinputlisting{Others/FastIO2.cpp}
\lstinputlisting{Others/FastIO3.cpp}
\subsection{取模}
\lstinputlisting{Others/ModInt.cpp}
\subsection{Int128}
\lstinputlisting{Others/int128.cpp}
\subsection{gp\_hash\_table}
\lstinputlisting{Others/gp_hash_table.cpp}
\subsection{关同步流}
\begin{lstlisting}
ios::sync_with_stdio(false);cin.tie(0);
\end{lstlisting}
\subsection{计时器}
\lstinputlisting{Others/Clock.cpp}
\end{document}
